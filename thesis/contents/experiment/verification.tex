\section{検証内容}
本実験では,様々な設定を変え,精度への影響を調査する.
検証内容と比較する条件設定は以下の通りである.

\begin{description}
    \item [検証1] AE特徴量の有無による分類精度の比較
    \begin{itemize}
        \item 全てのデータを学習させたオートエンコーダから抽出したAE特徴量を加える場合
        \item オートエンコーダから抽出したAE特徴量を加えない場合
    \end{itemize}

    \item [検証2] オートエンコーダの学習するクラスを変えた場合の分類精度の比較
    \begin{itemize}
        \item 全てのデータを学習させたオートエンコーダから抽出したAE特徴量を加える場合
        \item 多数派クラスのみを学習させたオートエンコーダから抽出したAE特徴量を加える場合
        \item 少数派クラスのみを学習させたオートエンコーダから抽出したAE特徴量を加える場合
    \end{itemize}

    \item [検証3] 異なるオートエンコーダの構成による精度の比較
    \begin{itemize}
        \item Encoder層の次元数を[入力データの次元数, 20, 10, 5]とした場合
        \item Encoder層の次元数を[入力データの次元数, 20, 15, 10, 5]とした場合
    \end{itemize}

    \item [検証4] AE特徴量の数による精度の比較
    \begin{itemize}
        \item AE特徴量の数を[入力データの次元数, 20, 10, 5]とした場合
        \item AE特徴量の数を[入力データの次元数, 20, 15, 10]とした場合
    \end{itemize}

    \item [検証5] 入力データの異なる前処理による分類精度の比較
    \begin{itemize}
        \item 入力データの前処理を行わず,オートエンコーダに学習させた場合
        \item 標準化した入力データをオートエンコーダに学習させ,AE特徴量は標準化しない場合
        \item 標準化した入力データをオートエンコーダに学習させ,E特徴量も標準化する場合
        \item 正規化した入力データをオートエンコーダに学習させ,AE特徴量は正規化しない場合
        \item 正規化した入力データをオートエンコーダに学習させ,AE特徴量も正規化する場合
    \end{itemize}

    \item [検証6] 異なる機械学習モデルによる分類精度の比較
    \begin{itemize}
        \item ロジスティック回帰を用いた場合
        \item ランダムフォレストを用いた場合
        \item SVMを用いた場合
        \item Multi Layer Perceptronを用いた場合
        \item LightGBMを用いた場合
    \end{itemize}

    \item [検証7] データセットごとに最も精度が高いモデルの比較

    \item [検証8] ハイパーパラメータチューニングを行なったモデルでの比較
    

\end{description}