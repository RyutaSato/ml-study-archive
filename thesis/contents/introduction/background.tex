\section{研究背景}

近年の多クラス分類タスクにおける教師あり機械学習は,非常に高い精度を達成している.\\
代表的なものとして,k-NNやSVM,ロジスティック回帰のような古くから提唱されてきたモデル,LightGBM\cite{lightgbm}やXGBoostといった勾配ブースティング決定木,そして近年では,特にニューラルネットワークを用いた深層学習モデルの研究が活発である.\\
しかし,教師あり学習全般に関わる問題として,データのラベル付けにコストがかかることが挙げられる.\\
教師データの作成時点では,データがどのクラスに属するかを人間が判断する必要がある.\\
さらに学習の精度向上には,データの量が重要であるため,データの量が多いほど精度は向上するが,コストも増加する.\\
しかし,不正アクセスなど稀にしか発生しないような事柄を収集したデータの場合,そもそものデータの量が少ないため,教師あり学習を行うことが難しい.\\
したがって,コスト面,データ量面での課題を解決できるような教師あり学習の手法が求められている.\\

