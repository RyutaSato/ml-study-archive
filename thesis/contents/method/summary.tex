\section{概要}
本研究では,オートエンコーダを用いた特徴量拡張に基づく新たなモデル構築手法を提案する.\cite{ae-article}
提案手法では,オートエンコーダを用いて入力データを学習する.学習済みのオートエンコーダのエンコーダの出力から入力データを低次元に圧縮した潜在表現を獲得できる.
この潜在表現を新たな特徴量として定義し,元のデータを拡張する.
この拡張されたデータを用いて,各種の教師あり学習モデル(ロジスティック回帰、勾配ブースティング決定木など)で学習を行う.
% 提案手法の概要を図\ref{fig:proposed_method}に示す.\\
\\
不均衡データで学習した機械学習モデルは,少数派クラスの特徴を十分に捉えられないことが分類精度の低下を招く.オートエンコーダを用いた特徴量拡張により,入力データの本質的な特性を別の表現として与えるができるため,機械学習モデルの判断材料を増加させ,サンプル数が少なくても,少数派クラスの特徴を捉えやすくなることが期待できる.\\

